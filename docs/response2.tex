\documentclass{amsart}

\usepackage{amsmath,amssymb,amsthm,mathtools}

\newcommand{\Z}{\mathbb{Z}}
\newcommand{\R}{\mathbb{R}}
\newcommand{\vol}{\operatorname{vol}}
\newcommand{\SL}{\mathrm{SL}}

\theoremstyle{definition}
\newtheorem{definition}{Definition}[section]
\theoremstyle{plain}
\newtheorem{theorem}[definition]{Theorem}
\newtheorem{lemma}[definition]{Lemma}
\newtheorem{proposition}[definition]{Proposition}

\begin{document}

\title{Unimodular Positive-Definite Ternary Quadratic Forms}
\author{}
\date{}
\maketitle

\section{Quadratic forms and equivalence}

\begin{definition}[Integral quadratic form, Gram matrix, determinant]
An \emph{integral ternary quadratic form} is a function $F:\Z^3\to \Z$ of the form
\[
F(x,y,z)=ax^2+by^2+cz^2+2dxy+2exz+2fyz
\quad(a,b,c,d,e,f\in\Z).
\]
Equivalently, there is a symmetric matrix $M_F\in M_3(\Z)$ such that
\[
F(\mathbf{x})=\mathbf{x}^T M_F \mathbf{x}\qquad(\mathbf{x}\in\Z^3).
\]
The \emph{determinant} of $F$ is $\Delta(F):=\det(M_F)$.
\end{definition}

\begin{definition}[Positive definite]
An integral quadratic form $F$ with Gram matrix $M_F$ is \emph{positive definite} if
$\mathbf{x}^T M_F \mathbf{x}>0$ for all $\mathbf{x}\in\R^3\setminus\{0\}$.
\end{definition}

\begin{definition}[Proper equivalence]
Two integral ternary quadratic forms $F$ and $G$ are \emph{properly equivalent} if there exists
$A\in \SL_3(\Z)$ such that
\[
M_G = A^T M_F A.
\]
\end{definition}

\section{Minkowski's theorem}

\begin{definition}[Centrally symmetric convex body]
A set $K\subset \R^n$ is a \emph{centrally symmetric convex body} if it is convex, measurable,
has nonempty interior, and satisfies $K=-K$.
\end{definition}

\begin{lemma}[Blichfeldt]\label{lem:blichfeldt}
Let $S\subset \R^n$ be measurable. If $\vol(S)>1$, then there exist distinct $x,y\in S$ with
$x-y\in \Z^n\setminus\{0\}$.
\end{lemma}

\begin{proof}
Let $Q=[0,1)^n$. For each $m\in\Z^n$, set $S_m=S\cap(m+Q)$, so the sets $S_m$ are pairwise disjoint and
$\vol(S)=\sum_{m\in\Z^n}\vol(S_m)$.

For each $m$, translate $S_m$ by $-m$ into $Q$ and denote the image by $T_m\subset Q$. Then $\vol(T_m)=\vol(S_m)$.
If all the sets $T_m$ were pairwise disjoint, then
\[
\sum_{m\in\Z^n}\vol(T_m) \le \vol(Q)=1,
\]
contradicting $\sum_m \vol(T_m)=\sum_m \vol(S_m)=\vol(S)>1$.
Hence $T_m\cap T_{m'}\neq\varnothing$ for some $m\neq m'$. Choose $u\in T_m\cap T_{m'}$, and let
$x=u+m\in S_m\subset S$ and $y=u+m'\in S_{m'}\subset S$. Then $x\neq y$ and $x-y=m-m'\in\Z^n\setminus\{0\}$.
\end{proof}

\begin{theorem}[Minkowski for $\Z^n$]\label{thm:minkowski}
Let $K\subset \R^n$ be a centrally symmetric convex body. If $\vol(K)>2^n$, then $K$ contains a nonzero point of $\Z^n$.
\end{theorem}

\begin{proof}
Set $S=\tfrac12 K$. Then $S$ is measurable and $\vol(S)=\vol(K)/2^n>1$.
By Lemma~\ref{lem:blichfeldt} there exist $x\neq y$ in $S$ with $m=x-y\in \Z^n\setminus\{0\}$.
Write $x=\tfrac12 u$ and $y=\tfrac12 v$ with $u,v\in K$. Then
\[
m=x-y=\tfrac12(u-v)=\tfrac12(u+(-v)).
\]
Since $K=-K$, we have $-v\in K$, and since $K$ is convex, $\tfrac12(u+(-v))\in K$.
Thus $m\in K\cap(\Z^n\setminus\{0\})$.
\end{proof}

\begin{lemma}\label{lem:pi_gt_3}
$\pi>3$.
\end{lemma}

\begin{proof}
In the unit circle, the regular inscribed hexagon has side length $1$ and perimeter $6$.
Each circular arc subtending a side is longer than the corresponding chord, so the circumference $2\pi$ exceeds $6$.
Thus $\pi>3$.
\end{proof}

\begin{lemma}[Volume of an ellipsoid]\label{lem:ellipsoidvol}
Let $M$ be a symmetric positive definite $n\times n$ real matrix, and let
\[
E_M(t)=\{\mathbf{x}\in\R^n : \mathbf{x}^T M \mathbf{x}\le t\}\qquad(t>0).
\]
Then
\[
\vol\bigl(E_M(t)\bigr)=\vol(B_n(1))\frac{t^{n/2}}{\sqrt{\det(M)}},
\]
where $B_n(1)=\{\mathbf{y}\in\R^n:\|\mathbf{y}\|\le 1\}$.
\end{lemma}

\begin{proof}
Choose an invertible $A$ with $M=A^T A$ (e.g.\ Cholesky). Then $\mathbf{x}^T M\mathbf{x}=\|A\mathbf{x}\|^2$ and
$E_M(t)=A^{-1}(B_n(\sqrt{t}))$.
By the change-of-variables formula,
\[
\vol\bigl(E_M(t)\bigr)=\frac{\vol(B_n(\sqrt{t}))}{|\det(A)|}
=\frac{\vol(B_n(1))\,t^{n/2}}{\sqrt{\det(A^T A)}}
=\vol(B_n(1))\frac{t^{n/2}}{\sqrt{\det(M)}}.
\]
\end{proof}

\section{Two lattice lemmas}

\begin{lemma}[A unimodular change of basis for a primitive vector]\label{lem:primitive_to_e1}
Let $v\in \Z^3$ be \emph{primitive}, i.e.\ $\gcd(v_1,v_2,v_3)=1$. Then there exists $U\in \SL_3(\Z)$ such that $Uv=e_1$.
Equivalently, there exists $A\in \SL_3(\Z)$ with $Ae_1=v$.
\end{lemma}

\begin{proof}
Write $v=(a,b,c)^T$ and set $d=\gcd(a,b)$. Choose integers $x,y$ with $xa+yb=d$.
Then the $2\times 2$ matrix
\[
P=\begin{pmatrix} x & y\\ -\frac{b}{d} & \frac{a}{d}\end{pmatrix}
\in \SL_2(\Z)
\]
satisfies $P(a,b)^T=(d,0)^T$. Define $U_1=\mathrm{diag}(P,1)\in \SL_3(\Z)$, so
\[
U_1 v = (d,0,c)^T.
\]
Since $\gcd(d,c)=1$, choose integers $r,s$ with $rd+sc=1$. Then
\[
Q=\begin{pmatrix} r & s\\ -c & d\end{pmatrix}\in \SL_2(\Z)
\]
satisfies $Q(d,c)^T=(1,0)^T$. Define $U_2\in \SL_3(\Z)$ acting by $Q$ on coordinates $(1,3)$ and fixing the second coordinate:
\[
U_2=\begin{pmatrix}
r & 0 & s\\
0 & 1 & 0\\
-c & 0 & d
\end{pmatrix}.
\]
Then $U_2(d,0,c)^T=(1,0,0)^T=e_1$. Hence $U=U_2U_1\in\SL_3(\Z)$ satisfies $Uv=e_1$.
Taking $A=U^{-1}$ gives $Ae_1=v$.
\end{proof}

\begin{lemma}[Shearing to remove a first row/column]\label{lem:shear}
Let $M=\begin{pmatrix}1 & r^T\\ r & C\end{pmatrix}$ be a symmetric $3\times 3$ matrix with $r\in\Z^2$ and $C\in M_2(\Z)$.
Let
\[
S=\begin{pmatrix}1 & -r^T\\ 0 & I_2\end{pmatrix}\in \SL_3(\Z).
\]
Then $S^T M S=\mathrm{diag}(1,\,C-r r^T)$.
\end{lemma}

\begin{proof}
Let $e_1,e_2,e_3$ be the standard basis. For $j=2,3$ one has $Se_1=e_1$ and $Se_j=e_j-r_{j-1}e_1$.
Thus
\[
(e_1)^T(S^T M S)e_j = (Se_1)^T M (Se_j) = e_1^T M (e_j-r_{j-1}e_1)= r_{j-1}-r_{j-1}\cdot 1=0.
\]
Hence $S^TMS$ has zero off-diagonal entries in the first row/column. The lower $2\times 2$ block equals
\[
(e_i-r_{i-1}e_1)^T M (e_j-r_{j-1}e_1)= C_{ij}-r_{i-1}r_{j-1}
\qquad(i,j\in\{2,3\}),
\]
which is $C-r r^T$.
\end{proof}

\section{Classification in determinant $1$}

\begin{lemma}[A unit vector in dimension $3$]\label{lem:unit3}
Let $F(\mathbf{x})=\mathbf{x}^T M\mathbf{x}$ be a positive-definite integral ternary quadratic form with $\det(M)=1$.
Then there exists $v\in\Z^3\setminus\{0\}$ with $F(v)=1$.
\end{lemma}

\begin{proof}
Let $E=\{\mathbf{x}\in\R^3:\mathbf{x}^T M\mathbf{x}\le 16/9\}$. By Lemma~\ref{lem:ellipsoidvol},
\[
\vol(E)=\vol(B_3(1))\frac{(16/9)^{3/2}}{\sqrt{\det(M)}}=\frac{4\pi}{3}\left(\frac{16}{9}\right)^{3/2}
=\frac{4\pi}{3}\left(\frac{4}{3}\right)^3=\frac{256\pi}{81}.
\]
By Lemma~\ref{lem:pi_gt_3}, $\vol(E)>\frac{256\cdot 3}{81}= \frac{256}{27}>8=2^3$.
By Minkowski's theorem (Theorem~\ref{thm:minkowski}) there exists $v\in \Z^3\setminus\{0\}$ with $v\in E$, i.e.\ $F(v)\le 16/9$.
Since $F(v)\in\Z$ and $16/9<2$, this forces $F(v)=1$.
\end{proof}

\begin{lemma}[A unit vector in dimension $2$]\label{lem:unit2}
Let $Q(\mathbf{y})=\mathbf{y}^T B\mathbf{y}$ be a positive-definite integral binary quadratic form with $\det(B)=1$.
Then there exists $u\in\Z^2\setminus\{0\}$ with $Q(u)=1$.
\end{lemma}

\begin{proof}
Let $E=\{\mathbf{y}\in\R^2:\mathbf{y}^T B\mathbf{y}\le 16/9\}$. By Lemma~\ref{lem:ellipsoidvol},
\[
\vol(E)=\vol(B_2(1))\frac{16/9}{\sqrt{\det(B)}}=\pi\cdot \frac{16}{9}.
\]
By Lemma~\ref{lem:pi_gt_3}, $\vol(E)>\frac{16\cdot 3}{9}=\frac{16}{3}>4=2^2$.
By Minkowski's theorem (Theorem~\ref{thm:minkowski}) there exists $u\in \Z^2\setminus\{0\}$ with $Q(u)\le 16/9$.
Since $Q(u)\in\Z$ and $16/9<2$, it follows that $Q(u)=1$.
\end{proof}

\begin{theorem}\label{thm:unimodular_ternary}
Every positive-definite integral ternary quadratic form $F$ with determinant $1$ is properly equivalent to
$X^2+Y^2+Z^2$.
\end{theorem}

\begin{proof}
Let $F(\mathbf{x})=\mathbf{x}^T M\mathbf{x}$ with $M\in M_3(\Z)$ symmetric, positive definite, and $\det(M)=1$.

By Lemma~\ref{lem:unit3} choose $v\in\Z^3$ with $F(v)=1$. Then $v$ is primitive.
By Lemma~\ref{lem:primitive_to_e1} choose $A_0\in \SL_3(\Z)$ with $A_0e_1=v$ and set
\[
M_0 = A_0^T M A_0.
\]
Then $(M_0)_{11}=e_1^T M_0 e_1 = v^T M v = F(v)=1$. Write
\[
M_0=\begin{pmatrix}1 & r^T\\ r & C\end{pmatrix}
\quad\text{with } r\in\Z^2,\ C\in M_2(\Z)\text{ symmetric}.
\]
Let $S_1\in\SL_3(\Z)$ be the shear from Lemma~\ref{lem:shear}. Then
\[
M_1 := S_1^T M_0 S_1 = \mathrm{diag}(1,\,B),\qquad B:=C-r r^T\in M_2(\Z)\text{ symmetric}.
\]
Since $\det(M_1)=\det(M_0)=\det(M)=1$, we have $\det(B)=1$, and $B$ is positive definite.

By Lemma~\ref{lem:unit2} there exists $u\in\Z^2$ with $u^TBu=1$.
Choose $A_2\in \SL_2(\Z)$ with $A_2 e_1=u$ (by the $2$-dimensional version of Lemma~\ref{lem:primitive_to_e1}),
and set $\widetilde{A}_2=\mathrm{diag}(1,A_2)\in \SL_3(\Z)$. Then
\[
M_2 := \widetilde{A}_2^T M_1 \widetilde{A}_2 = \mathrm{diag}(1,\,B_2),
\qquad B_2:=A_2^T B A_2,
\]
and $(B_2)_{11}=u^T B u=1$. Write
\[
B_2=\begin{pmatrix}1 & s\\ s & t\end{pmatrix}\qquad(s,t\in\Z).
\]
Let $S_2=\begin{pmatrix}1 & -s\\ 0 & 1\end{pmatrix}\in \SL_2(\Z)$ and $\widetilde{S}_2=\mathrm{diag}(1,S_2)\in\SL_3(\Z)$.
A direct calculation (the $2\times 2$ analogue of Lemma~\ref{lem:shear}) gives
\[
S_2^T B_2 S_2 = \begin{pmatrix}1 & 0\\ 0 & t-s^2\end{pmatrix}.
\]
Thus
\[
M_3 := \widetilde{S}_2^T M_2 \widetilde{S}_2 = \mathrm{diag}\bigl(1,1,t-s^2\bigr).
\]
Taking determinants yields $1=\det(M_3)=t-s^2$, so $M_3=I_3$.

Therefore
\[
I_3 = M_3 = (A_0 S_1 \widetilde{A}_2 \widetilde{S}_2)^T\, M\, (A_0 S_1 \widetilde{A}_2 \widetilde{S}_2),
\]
and the product $A_0 S_1 \widetilde{A}_2 \widetilde{S}_2$ lies in $\SL_3(\Z)$.
Hence $F$ is properly equivalent to $X^2+Y^2+Z^2$.
\end{proof}

\end{document}
